\documentclass[a4,12pt]{paper}

\usepackage[brazil,portuguese,english]{babel}
\usepackage[utf8]{inputenc}
\usepackage[T1]{fontenc}
\usepackage{setspace}
\usepackage{mathtools}
\usepackage{graphicx}
\usepackage{lscape}
\usepackage{verbatim}
\usepackage[colorlinks,linkcolor=black,citecolor=black,urlcolor=black,hyperindex]{hyperref}
\usepackage[lined,boxed,commentsnumbered]{algorithm2e}
\usepackage{amssymb}

\pagestyle{headings}
\setlength{\textwidth}{16cm}
\setlength{\textheight}{21cm}
\setlength{\oddsidemargin}{5mm}
\setlength{\evensidemargin}{5mm}
\setlength{\topmargin}{1cm}
\setlength{\footskip}{1cm}
\setlength{\headsep}{1cm}
\setlength{\headheight}{0cm}
\bibliographystyle{plain}

\hyphenation{na-tu-rais}
\hyphenation{cons-tru-í-do}
\hyphenation{pro-ba-bi-li-da-des}
\hyphenation{si-mi-la-ri-da-de}
\hyphenation{pa-ra-le-lis-mo}
\hyphenation{eu-cli-dia-na}
\hyphenation{cons-tru-í-do}
\hyphenation{pro-ba-bi-li-da-des}
\hyphenation{mi-ni-mi-zar}
\hyphenation{des-cri-tos}
\hyphenation{li-mia-ri-za-ção}
\hyphenation{inú-me-ras}
\hyphenation{ima-gem}
\hyphenation{res-pecti-vas}

\begin{document}

\thispagestyle{empty}
\begin{center}

\huge{Trabalho de Conclusão de Curso}\\[3 cm]

\large{Autores\\
Denis Henrique Nunes\\
Gabriel Faquini\\
Leandro Carlos Rodrigues\\
Wagner Barbosa
}\\[2 cm]


\huge{\textbf{Contagem Automática de Marcações em Células Microscópicas}}\\[4 cm]

\large{Orientador: Paulo Sérgio Rodrigues}\\
\large{Co-orientador: Rodrigo Filev}\\
\large{Instituição: Centro Universitário da FEI}\\[2 cm]


\large{São Bernardo do Campo, SP}\\
\large{Março de 2013}
\end{center}
\newpage

\pagestyle{empty}
%\tableofcontents
\newpage

\pagestyle{headings}
\onehalfspacing

\section*{Resumo}

Pesquisadores de várias áreas da biologia utilizam métodos de contagem de certas estruturas biológicas, relevantes para suas pesquisas científicas. Na área da neurociência, são realizadas marcações da proteína DeltaFosB, que é encontrada no tecido nervoso. Tal proteína parece estar relacionada em fenômenos de longa duração como dependências de drogas e crises epiléticas crônicas.

Para estudar uma dependência química ou crises epiléticas em um animal, é utilizada uma marcação específica para esta proteína e realisa-se uma contagem manual da estrutura marcada.

A contagem de certo grupo de animais pode levar semanas. Existem softwares que auxiliam nesta contagem, mas ainda são dependentes de monitoramento humano. Além disto, a quantidade de erros da contagem manual, ou semi-manual, é diretamente proporcional ao cansaço da pessoa que o realiza.

Este trabalho tem como objetivo propor técnicas capazes de dar autonomia completa para um sistemas de contagem de estruturas biológicas marcadas, através de aprendizado por reforço.

Pretende-se solucionar o problema utilizando várias técnicas de visão computacional e inteligência artificial, através de segmentação por detecção de borda e classificação via redes neurais.


\end{document}
